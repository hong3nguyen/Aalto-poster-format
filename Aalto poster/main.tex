% This work is inspired by \title{Aalto poster} \author{Juho Rousu}
% I must suggest you take a look on Juho Rousu poster for a good view of various logo options.
\documentclass[landscape,a0,final]{a0poster} % a0poster class sets the paper size
%\pdfobjcompresslevel=2
\usepackage[utf8]{inputenc}
\usepackage[english]{babel}
\usepackage[pdftex]{graphicx}
%s\usepackage[SCI,RGB]{aaltologo} % RGB, Coated, Uncoated / logo here
%\usepackage{lipsum} % Lorem ipsum generator

\usepackage{titlesec} % For changing the font on chapters, sections, etc.

%\usepackage{titlesec} % For changing the font on chapters, sections, etc.
\usepackage{float} % Improved interface for floating objects



\usepackage{epstopdf} % so that ``pdflatex'' accepts .eps files

\usepackage{todonotes}
% Teach hyphenation for alien words
\hyphenation{op-tical net-works semi-conduc-tor}

%\newcommand{\vt}[1]{\ensuremath{\boldsymbol{#1}}} % vector, i.e. math bold face, cursive
\newcommand{\vt}[1]{\ensuremath{\mathbf{#1}}} % vector, i.e. math bold face, not-cursive
\newcommand{\lt}[1]{\ensuremath{\mathrm{#1}}} % sub or sup text, i.e. not cursive
\newcommand{\ec}{\ensuremath{\vt{J}}} % electric surface current
\newcommand{\mc}{\ensuremath{\vt{M}}} % magnetic surface current
%
\newcommand{\aee}{\ensuremath{\overline{\overline{a}}_\lt{ee}}}
\newcommand{\aem}{\ensuremath{\overline{\overline{a}}_\lt{em}}}
\newcommand{\ame}{\ensuremath{\overline{\overline{a}}_\lt{me}}}
\newcommand{\amm}{\ensuremath{\overline{\overline{a}}_\lt{mm}}}
\newcommand{\bee}{\ensuremath{\overline{\overline{b}}_\lt{ee}}}
\newcommand{\bem}{\ensuremath{\overline{\overline{b}}_\lt{em}}}
\newcommand{\bme}{\ensuremath{\overline{\overline{b}}_\lt{me}}}
\newcommand{\bmm}{\ensuremath{\overline{\overline{b}}_\lt{mm}}}
%
\newcommand{\adyad}{\ensuremath{\overline{\overline{a}}}}
\newcommand{\bdyad}{\ensuremath{\overline{\overline{b}}}}
%
\newcommand{\unitx}{\ensuremath{\vt{x}_0}}
\newcommand{\unity}{\ensuremath{\vt{y}_0}}

\newcommand{\sectionspace}{10mm} % Free space before each section inside a minipage
\newcommand{\figurespace}{10mm} % Free space around figures inside a minipage (where floats are not allowed)

\usepackage[skins]{tcolorbox}
\newtcolorbox{myframe}[2][]{%
  enhanced,colback=white,colframe=black,coltitle=black,
  sharp corners,boxrule=0.7pt,
  fonttitle=\itshape,
  attach boxed title to top left={yshift=-0.3\baselineskip-0.4pt,xshift=2mm},
  boxed title style={tile,size=minimal,left=0.5mm,right=0.5mm,
    colback=white,before upper=\strut},
  title=#2,#1
}

\begin{document}

\thispagestyle{empty} % Removes the page number


\begin{minipage}[t]{0.98\linewidth} % The first minipage for the logo & title
\vspace{0pt} % A trick to align the parallel minipages on top

\vspace{0.008\linewidth} % Increase the top margin

\begin{minipage}[t]{0.09\linewidth} % logo
\vspace{0pt} % Alingns the parallel minipages on top

%% Choose the logo or use random generator
%\AaltoLogoLarge{1.55}{''}{aaltoBlue} % Chosen logo, scaled for A1 size
%\AaltoLogoRandomLarge{1.55} % Random logo, scaled for A1 size
%\AaltoLogoLarge{1.55}{''}{aaltoBlack}

\end{minipage} % no empty line before the next begin
\begin{minipage}[t]{0.72\linewidth} % title
\vspace{0pt} % Alingns the parallel minipages on top


% More conservative title, upright and black
{\renewcommand{\baselinestretch}{0.85} % Changes the baseline skip smaller for the title
\Huge{\bfseries{\textsf{Service Engineering for ML Systems, Edge-Cloud Continuum, Big Data, and IoT}}} % Text size for a1 posters
\par} % <- for \baselinestretch
\vspace{0.01\linewidth} % Empty space after the title

\normalsize{\textsf{\bfseries{Hong-Tri Nguyen, Minh-Tri Nguyen, Debayan Bhattacharya, Anh-Dung Nguyen, Viet-Chau Nguyen, Hong-Linh Truong}}} % Text size for a1 posters

\textcolor{aaltoGray}{\textsf{\bfseries{Aalto Systems and Services Engineering Analytics (AaltoSEA) Group, Department of Computer Science, Aalto University}}}
%
\end{minipage}
\begin{minipage}[t]{0.35\linewidth}
\vspace{-10pt} % A trick to align the parallel minipages on top

\vspace{0.008\linewidth} % Increase the top margin
\includegraphics[width=0.5\linewidth]{Aalto poster/figs/qrcode-AaltoSEA.pdf}
% \end{center}
\end{minipage}
\end{minipage}

%% Two columns

% Space according to the visual identity guidelines...
\vspace{0.005\linewidth}

% Centering helps in placement
\centering

\small % Text size for a1 posters

\begin{minipage}{0.98\linewidth}

\begin{minipage}[t]{0.3\linewidth}
\setlength{\parindent}{10mm} % Paragraph indent

\vspace{\sectionspace}


\section{QoA4ML -- Quality-of-Analytics for Machine Learning Library}
%\vspace{\figurespace}

\begin{center}
    
\includegraphics[width=0.9\linewidth]{./figs/QoA4ML-QR.pdf}
\end{center}
\begin{myframe}{Motivation}
    Which ML-specific attributes and constraints should be specified and assured in service contracts for ML services, and how can they be effectively communicated and monitored among relevant stakeholders?
\end{myframe}

\begin{myframe}{Features}
    QoA4ML as a monitoring library to support ML-specific attributes in ML contracts.
    \begin{itemize}
        \item An ML-specific specification in the service contract for ML applications.
        \item Monitoring utilities/probes for ML-specific metrics and data quality.
        \item Observation agents for evaluating runtime quality, enhancing runtime explainability for end-to-end ML serving.
    \end{itemize}



\end{myframe}
\section{Analytics of Programming and Service Models for Hybrid Quantum Computing Software}

\begin{center}

\includegraphics[width=\linewidth]{./figs/debayan2.pdf}
\end{center}

\begin{myframe}{Research questions}
    \begin{itemize}
    \item What methodologies can be developed to quantitatively measure and evaluate the performance metrics of hybrid classical-quantum computing systems?
    \item How can the analysis and optimization of service components within these systems enhance performance and achieve optimal resource balancing?
    \item Is it feasible to develop an analytical tool or prototype that provides real-time performance metrics and diagnostics of individual services impacting hybrid quantum application execution?
    \end{itemize}
 
\end{myframe}

\begin{myframe}{Key features}
    \begin{itemize}
        \item Performance Metrics for hybrid classical-quantum computing systems.
        \item Analytical tool for real-time monitoring.
        \item Intelligent quantum workflow management.
    \end{itemize}


\end{myframe}

\begin{myframe}{Project implications}
    \begin{itemize}
        \item Will help quantum researchers to decide on the commercial offerings.
        \item Help in continual quantum software development.
        \item Cloud-based workload management, more scalable architecture.
        \item Paving the way for Q-SWE industry standards.
        \item Service-oriented approach for quantum computing.
        
    \end{itemize}


\end{myframe}

% The end of the first column and the start of the second
\end{minipage} % no empty line before the next ``\begin''
\hspace{0.03\linewidth} % Middle margin
\begin{minipage}[t]{0.3\linewidth}
\setlength{\parindent}{10mm} % Paragraph indent
\vspace{\sectionspace}
\section{ROHE -- End-to-End ML Orchestration in Edge-Cloud Continuum}
%\vspace{-70pt}
\begin{center}
\includegraphics[width=\linewidth]{./figs/ROHE-QR.pdf}
\end{center}
\begin{myframe}{Research questions}
    How can resource be optimized for deploying end-to-end ML applications on heterogeneous edge environments to ensure performance and compliance with data regulations?
\end{myframe}

\begin{myframe}{Key findings}
\begin{itemize}
    \item Presented an orchestration framework to optimize resource utilization for end-to-end ML applications on heterogeneous edge environments.
    \item The framework involves profiling all microservices within the application to estimate scales and allocate them on suitable hardware platforms based on their runtime utilization patterns.
    \item Provided practical analyses on runtime monitoring metrics to detect and mitigate resource contentions, ensuring consistent performance.
\end{itemize}
\end{myframe}

\subsection*{Experimental results}

\begin{center}
\includegraphics[width=0.8\linewidth]{./figs/late_res.pdf}
\end{center}
\vspace{-20pt}
\subsection*{Current work}
\includegraphics[width=\linewidth]{./figs/ROHE-pro.pdf}
%\vspace{\figurespace}
\begin{myframe}{Research questions}
    How can the ROHE framework be optimized to balance prediction performance and service costs across diverse contexts such as network topology, model and service explainability, and energy-restricted environments?
\end{myframe}

\end{minipage}
\hspace{0.03\linewidth} % Middle margin
\begin{minipage}[t]{0.3\linewidth}
\setlength{\parindent}{10mm} % Paragraph indent
\vspace{\sectionspace}
\section{RXOMS -- Runtime eXplainability for Orchestrating ML-based Security Systems}
\vspace{-30pt}
\begin{center}
\includegraphics[width=0.9\linewidth]{./figs/RXOM-RQ.pdf}
\end{center}
\begin{myframe}{Research questions}
  How can we enhance the explainability and integration of ML-based security services within Digital Twins (DTs) to improve security operations in heterogeneous and distributed smart systems?
\end{myframe}
\begin{myframe}{Key findings}
\begin{itemize}
    \item Introduced a novel framework that unifies runtime contexts for better integration and interpretation of security services in DTs.
    \item Developed a workflow and necessary services to generate security reports across physical and logical layers.
    \item Utilized a centralized knowledge service to allow security analysts to incorporate domain knowledge for automating incident reasoning and security enforcement.
\end{itemize}
\end{myframe}

\subsection*{Current work}
\vspace{-20pt}
\begin{center}
    \includegraphics[width=0.95\linewidth]{Aalto poster/figs/AIagent-COMPSAC-23.pdf}
\end{center}
\begin{myframe}{Research questions}
  How leveraging language model agents can enhance the analysis of anomaly alerts in DT environments, enhancing the automated generation of anomaly reports based on current runtime contexts?
\end{myframe}
\section{Testbed}
\vspace{-20pt}
\begin{center}
\includegraphics[width=0.7\linewidth]{./figs/image.jpeg}
\end{center}

\begin{myframe}{Clusters}
    \begin{itemize}
        \item \textbf{Raspberry Pi Cluster:} Pi 4B/5 with Hailo/Coral AI accelerator
        \item \textbf{x86 Cluster:} Beelink BT3, HP Z240, and DellPrecision 5820 Tower 
        \item \textbf{Nvidia GPU Cluster:} Jetson Orin/AGX Xavier/Nano
    \end{itemize}
\end{myframe}

\end{minipage}
\end{minipage} % minipage for the two columns (minipages)


\vspace*{0.01\linewidth} % Increase the bottom margins

\end{document}
